\documentclass[a4paper,10pt]{scrreprt}
\usepackage[utf8x]{inputenc}

\begin{document}

\section*{Task agreement for CDT refactoring semester thesis}

During this semester thesis it is planned to introduce and improve one or more
refactorings to the Eclipse CDT project. The priorities are as follows

\begin{enumerate}
\item Toggle (Member-) Function Implementation
\item Re-implement Implement Function
\item Override virtual Member Function
\end{enumerate}

Our goal is to start with the first refactoring to gain insight on how to solve
the other two refactorings since they share some similarities. However it is not
our goal to implement as many refactorings as possible. We don't want our work
to be thrown away at the end of the semester. To become part of the main CDT
repository our work needs to fulfill three requirements: Quality, speed and
integration. 

Quality is achieved through automatic testing. Refactorings are thankful to test since they have a clear before and after state. As for speed, a maximum of one second of response time for small and frequent tasks is acceptable\footnote{Shneiderman, B., \emph{Response time and display rate in human performance with computers}, ACM Computing Surveys (CSUR), 1984}. For example \texttt{Extract function} consumes about four seconds to extract the body of a hello world code. Our goal is to process local changes in less than a second. The last matter is integration. We think we did good work if a first semester student sitting in front of a fresh CDT installation is able to install our refactoring in less than five minutes given the update site address. In addition the deployment process should take at most three commands to integrate a code change. Let's now turn to functional requirements. \newline

% To have a realistic chance to get a good mark we need SMART goals.
% SMART is an acronym for „Specific Measurable Accepted Realistic Timely“
There are theoretically six different ways of toggling function implementation
although not every variation is useful. Three of them are needed to toggle
circularly from \texttt{in-class} to \texttt{in-header} to
\texttt{separate-file}  and back again to \texttt{in-class}. We concentrate on
this order during this project.\newline
Depending on the success of the implementation of those three variations of the
first refactoring we continue with the re-implementation of  ``Implement
Function'' and then ``Override virtual member function''.

\pagebreak

\section*{General Goals}

\begin{itemize}
 \item Quality: All special cases are covered with test cases for each
refactoring subtype.
 \item Integration: Sitting in front of a fresh Eclipse CDT installation a first
semester student can install our refactoring in less than five minutes.
 \item Automation: After changing some code a maximum of three commands are
needed to redeploy the project.
\end{itemize}

\section*{Minimum Goals}

All general goals have been archived. Additionally:
\begin{itemize}
 \item Toggling between \texttt{in-class}, \texttt{in-header},
\texttt{separate-file} and back again to \texttt{in-class} works.
\end{itemize}

\section*{Medium Goals}
All minimum goals have been archived. Additionally:
\begin{itemize}
 \item Re-Implement the ``Implement Function'' feature.
 \item The ``Implement Function'' should work fast. even on a reasonably
slow machine. %define what is slow?
%more quality functions here
\end{itemize}

\section*{Maximum Goals}
All medium goals have been archived. Additionally:
\begin{itemize}
 \item ``Override virtual member function`` has been implemented.
 \item ''Override virtual member function`` works fast.\newline
\end{itemize}

The semester thesis starts on September 20th and has to be finished until
December 23rd, 2010.

\subsection*{Students}
Thomas Kallenberg \dotfill
Martin Schwab \dotfill
\subsection*{Project supervisor}
Peter Sommerlad \dotfill
~ \newline 
\end{document}

