\begin{abstract}

\begin{figure}
  \centering

  \begin{sequencediagram}
    \newthread{ss}{}{SimulationServer}
    \newinst{ctr}{}{SimControlNode}
    \newinst{ps}{}{PhysicsServer}
    \newinst[1]{sense}{}{SenseServer}

    \begin{call}{ss}{Initialize()}{sense}{}
    \end{call}
    \begin{sdloop}{Run Loop}
      \begin{call}{ss}{StartCycle()}{ctr}{}
        \begin{call}{ctr}{ActAgent()}{sense}{}
        \end{call}
      \end{call}
      \begin{call}{ss}{Update()}{ps}{}
        \begin{call}{ps}{PrePhysicsUpdate()}{sense}{state}
        \end{call}
        \begin{callself}{ps}{PhysicsUpdate()}{}
        \end{callself}
        \begin{call}{ps}{PostPhysicsUpdate()}{sense}{}
        \end{call}
      \end{call}
      \begin{call}{ss}{EndCycle()}{ctr}{}
        \begin{call}{ctr}{SenseAgent()}{sense}{}
        \end{call}
      \end{call}
    \end{sdloop}
  \end{sequencediagram}

  \caption{UML sequence diagram demo.}
\end{figure}

\begin{figure}
  \centering

  \begin{sequencediagram}
    \newthread{ss}{}{SimulationServer}
    \newinst{ps}{}{PhysicsServer}
    \newinst[1]{sense}{}{SenseServer}
    \newthread[red]{ctr}{}{SimControlNode}

    \begin{sdloop}[green!20]{Run Loop}
      \mess{ctr}{StartCycle}{ss}
      \begin{call}{ss}{Update()}{ps}{}
        \prelevel
        \begin{callself}{ctr}{SenseAgent()}{}
          \begin{call}[3]{ctr}{Read}{sense}{}
          \end{call}
        \end{callself}
        \prelevel\prelevel\prelevel\prelevel
        \setthreadbias{west}
        \begin{call}{ps}{PrePhysicsUpdate()}{sense}{}
        \end{call}
        \setthreadbias{center}
        \begin{callself}{ps}{Update()}{}
        \end{callself}
        \begin{call}{ps}{PostPhysicsUpdate()}{sense}{}
        \end{call}
      \end{call}
      \mess{ss}{EndCycle}{ctr}
      \begin{callself}{ctr}{ActAgent()}{}
        \begin{call}{ctr}{Write}{sense}{}
        \end{call}
      \end{callself}
    \end{sdloop}

  \end{sequencediagram}

  \caption{Example of a sequence with parallel activities.}
\end{figure}



During this semester thesis, a code automation tool has been developed for the 
Eclipse C++ Development Toolkit (CDT) using the Eclipse refactoring mechanism. 
The resulting plugin enables a C++ developer to move function definitions easily 
between header and source files.

The new plugin differs from existing plugins in the way that it minimizes human 
interaction by providing a single keystroke interface. The refactoring gets by 
with no user wizard at all and is tolerant to imprecise code selection. 

This document discusses the uses of the plugin as well as the issues that had 
to be handled with during the project. Students developing a new refactoring may 
have a look at the problems section to be able to start with their own project 
quickly. Project setup hints are listed in the appendix.
\end{abstract}

\chapter*{Management Summary}

The C++ specification allows functions to be written at different places inside source code. This may be used to separate interfaces from implementation. A header file that lists the declarations and a source file that holds the actual definitions. As functions grow in size it may become time to move implementation details into a separate file. This task involves copy and paste, changing the function signature and tedious and involves changing  files.

Solution

How it looks like?



This semester thesis tries to make programming C++ more fun and more efficient, 
providing a tool that automates code changes with a single keystroke. Using the 
developed plugin, a programmer can save touching multiple code files with just 
one action.

As in writing books, writing software involves text editors. Developers of 
computer software are using highly specialized text editors. These provide a 
wide range of tools that simplify daily tasks of a programmer. One such task is 
moving a function from one place to another. During this semester thesis, an 
existing C++ code editor has been extended to support moving functions between 
various places quick and user-friendly.
\thispagestyle{empty}
\pagebreak

\chapter*{Thanks}
We would not have been able to achieve this project without the help of others.
A big thanks goes to all of these people.\newline
Specially we would like to thank Prof. Peter Sommerlad for the original idea
of the toggle refactoring, as our supervisor of the project and for various cool
ideas in many problems we encountered. Lukas Felber who provided us with
instant solutions where we struggled to continue. Emanuel Graf for his ideas and
explanation in every subtopic of the CDT project. Thomas Corbat for help and
ideas for various subtopics like comment handling. \newline
Another big thank goes to our families and friends who were neglected during our
semester thesis.
\thispagestyle{empty}

