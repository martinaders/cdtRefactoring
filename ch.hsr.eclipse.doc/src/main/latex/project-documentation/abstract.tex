\begin{abstract}

During this semester thesis, a code automation tool has been developed for the 
Eclipse C++ Development Toolkit (CDT) using the Eclipse refactoring mechanism. 
The resulting plugin enables a C++ developer to move function definitions easily 
between header and source files.

The new plugin differs from existing plugins in the way that it minimizes human 
interaction by providing a single keystroke interface. The refactoring gets by 
with no user wizard at all and is tolerant to imprecise code selection. 

This document discusses the uses of the plugin as well as the issues that had 
to be handled with during the project. Students developing a new refactoring may 
have a look at the problems section to be able to start with their own project 
quickly. Project setup hints are listed in the appendix.
\end{abstract}

\chapter*{Management Summary}
In C++ there is the possibility to tell the compiler there exists a function
with a so called \textit{declaration}. Since
this does not specify what the function does, a \textit{definition} is needed
with the functionality.

\begin{lstlisting}[caption={class with declaration and definition},
label={declanddef}, language=C++]
class A {
  int function(int param1); //declaration
};

inline int function(int param1)
{
  return param1 + 23; //definition
}
\end{lstlisting}

Every definition is a declaration too.

\begin{lstlisting}[caption={class with declaration and definition},
label={defonly}, language=C++]
class A {
  int function(int param1) { //decl. and def.
    return param1 += 23;
  }
};
\end{lstlisting}

Since a declaration and definition can be separated differences can occur
between the signature of the definition and the signature of the declaration.
This is not allowed. 
Imagine now that changing a function signature in C++ is an unthankful task.
As an additional barrier, a declaration and a definition can occur in two
different files.

\vspace{0.5cm}
\begin{minipage}{.48\textwidth}
\lstset{xrightmargin=0.5cm}
\begin{lstlisting}[caption={Header file containing declaration},
label={twofile1}, language=C++]

class A {
  void foo();
};

#endif /* A_H_ */
\end{lstlisting}
\end{minipage}%
\begin{minipage}{.48\textwidth}
\lstset{xleftmargin=0.5cm}
\begin{lstlisting}[caption={Source file containing definition},
label={twofile2}, language=C++ ]
#include "A.h"

void A::foo() {
  return
}
\end{lstlisting}
\end{minipage}

If a signature changes, these changes must be made in two files, the header file
and the implementation file. More than once programmers forget to change the
signature in one place which results in compile errors and unnecessary time
consuming error correction.\newline

Refactorings do solve such problems by automating changes to source code so,
less errors are introduced by hand.

The ``One touch toggle refactoring'' moves function definition in
\textit{Eclipse} from one position to another and preserves correctness.

This is done by searching for a function definition or declaration next to the
cursor position or the selection of source code. Then, according to the found
element, it's sibling is searched. Then the signature of the definition is
copied and adapted to the new position. The new definition gets inserted and
the old definition is removed.
If the definition and the declaration is the same in the beginning, the old
definition is replaced with a newly created declaration.

All this is done without any wizard and kept speedy to not break the work flow.
The refactoring is bound to a key.
\thispagestyle{empty}
\pagebreak

\chapter*{Thanks}
We would not have been able to achieve this project without the help of others.
A big thanks goes to all of these people.\newline
Specially we would like to thank Prof. Peter Sommerlad for the original idea
of the toggle refactoring, as our supervisor of the project and for various cool
ideas in many problems we encountered. Lukas Felber who provided us with
instant solutions where we struggled to continue. Emanuel Graf for his ideas and
explanation in every subtopic of the CDT project. Thomas Corbat for help and
ideas for various subtopics like comment handling. \newline
Another big thank goes to our families and friends who were neglected during our
semester thesis.
\thispagestyle{empty}

