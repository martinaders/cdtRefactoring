\chapter{Implementation and Solution}
\thispagestyle{fancy}

From the three specified refactorings, \textit{Toggle Function Definition} has 
been implemented in depth. This chapter explains how the refactoring was 
implemented and which problems had to be solved during development.

\section{Architecture}

In Eclipse, most of the architecture is already given. Some specialities of the 
toggle refactoring are presented in this section.

\subsubsection{Basic call flow}
Here's the place for some nice diagram to visualize the work.

Activator -- ActionDelegate -- (Job) -- ToggleRefactoring -- Strategy

\subsubsection{The strategies}
The \textit{ToggleStrategyFactory} is used to decide which strategy should be 
considered, based on the user selection. The following strategies have been 
implemented to cover the specified cases:

\begin{itemize}
\item ToggleFromImplementationToClassStrategy
\item ToggleFreeFunctionFromInHeaderToImpl
\item ToggleFromClassToInHeaderStrategy
\item ToggleFromInHeaderToClassStrategy
\item ToggleFromInHeaderToImplementationStrategy
\end{itemize}

\subsubsection{implications of not using a refactoring wizard}
No wizard was used for this refactoring since it must be fast and may be 
executed several times in succession. When using a wizard, the 
\textit{RefactoringWizardOpenOperation} handles the execution of the refactoring 
inside a separate job. Since the toggle refactoring does not use the wizard, a 
separate job had to be scheduled by the ActionDelegate.

In addition, the undo functionality had to be implemented separately. When the 
changes are performed, they (surprisingly) also return the undo changes that are 
needed by the UndoManager. The UndoManager is available through 
\textit{RefactoringCore.getUndoManager()}. See \textit{ToggleRefactoringRunner} 
for more details on the implementation.

\subsection{Issues}

This section describes special cases that have been omitted intentionally or due 
to lack of time. In addition, found limitations of the CDT are described here.

\subsubsection{Speed}
\textbf{Problem}: Refactoring, especially the first run, was very slow in the 
beginning. Including a big library slowed down the process even more.
\textbf{Cause}: The first thought was that header file indexing was the cause. 
However, the indexer option that skips already indexed headers is enabled in 
\textit{CRefactoring}. In the end, it was found out that most of the time was 
consumed by the \textit{checkInitialConditions} method of \textit{CRefactoring} 
that checked for problems inside the translation unit.
\textbf{Solution}: The super call to \textit{checkInitialConditions} was omitted.

\subsubsection{Accessing standalone header files}
\textbf{Problem}: Header files that are not included in any source file by 
default were not found by the indexer. Thus, it was not possible to analyze the 
source code of the affected header file.
\textbf{Cause}: By default, the indexer preference option 
\textit{IndexerPreferences.KEY\_INDEX\_UNUSED\_HEADERS\_WITH\_DEFAULT\_LANG} is 
set to false. However, this option is needed for standalone header files to be 
indexed.
\textbf{Solution}: Set the described option in \textit{IndexerPreferences} to 
true.

\subsubsection{Handling newly created files}
\textbf{Problem}: It is difficult do manipulate newly created files.
\textbf{Cause}: --
\textbf{Solution}: --

\subsubsection{Constructor / destructor bug}
\textbf{Problem}: Let CDT create a new class with a constructor and a destructor. 
Then toggle the constructor out of the class definition. The Destructor will be 
overridden partially. This problem only occurs in exactly this situation (no 
paramenters, no initialization lists).
\textbf{Cause}: Unknown. It seems to be some offset bug.
\textbf{Solution}: None yet solved.

\subsubsection{Unneccessary newlines}
\textbf{Problem}: When toggling multiple times, a lot of newlines are generated 
by the rewriter.
\textbf{Cause}: The rewriter inserts newlines before and after inserted nodes 
but does not remove them when the same node is removed.
\textbf{Solution}: No satisfying solution was found. The formatter may be used 
to remove multiple newlines. It was tried to manually change the generated text 
changes to avoid inserting or delete more newlines. However, this solution is 
highly dependent on the actual changes that are made. In addition, the generated 
arrays of changes are not guaranteed to have the same changes at the same index 
all the time (and yes, they do change). The resulting code was unstalbe and this 
solution is not recommended. Another point is, that the situation has to be 
evaluated carefully since it is not always appropriate not to insert or remove 
all generated and existing newlines.

\subsubsection{Menu integration}
\textbf{Problem}: Adding a new menu item to the refactor menu is difficult when 
developing a separate plugin.
\textbf{Cause}: Menu items are hardcoded inside 
\textit{CRefactoringActionGroup}. No way was found to replace or change this 
class within a separate plugin. In addition, the use of the 
\textit{org.eclipse.ui.actionSets} extension point does not make inserting new 
items easier.
\textbf{Solution}: The menu was added using \textit{plugin.xml} and may be added 
by the user manually. Right-click the toolbar, choose "Customize Perspective...", 
"Command Groups Availability" and check every group that is named "C++ Coding". 
This reveals the new menu item inside the refactor menu. Anyhow, the refactoring 
may always be invoked using the key binding.

The toggle key binding was realized using the \textit{org.eclipse.ui.bindings} 
extension point.

\subsubsection{The selection}
\textbf{Problem}: After toggling multiple times, the wrong functions were 
toggled or no selected function was found at all. 
\textbf{Cause}: The region provided by \textit{CRefactoring} pointed to a wrong 
code offset. 
\textbf{Solution}: The current selection is now based directly on the current 
active editor part's selection.

\subsubsection{Comments and macros}
\textbf{Problem}: Nodes inside a translation unit have to be copied to be 
changed since they are frozen. When nodes are copied, their surrounding comments 
get lost during rewrite. This was annoying, since copying the function body 
provided a straightforward solution for replacing a declaration with a 
definition.

Another issue were macros. Macros are working perfectly when copied and 
rewritten inside the same translation unit. As soon as a macro is moved outside 
to another translation unit, the macro will be expanded during rewrite. 

\textbf{Cause}: The rewriter is using a method in \texttt{ASTCommenter} to get a 
\texttt{NodeCommentMap} of the rewritten translation unit. If a node is copied, 
it has another reference which won't be inside the comment map anymore. Thus, 
when the rewriter writes the new node, it won't notice that the node was 
replaced by another.

\textbf{Solutions}:
\begin{itemize}
\item Get the raw signature of the code parts that should be copied and insert 
them using an ASTLiteralNode. 

Pro: It works without changing the CDT core and macros are not expanded. 

Contra: Breaks indentation and inserts unneeded newlines. This solution was used 
finally because whitespace issues may be dealt with the formatter.
\item Do as \texttt{ExtractFunction} does: rewrite each statement inside the 
function body separately. 

Pro: automatic indentation. 

Contra: touches the body although it does not need to be changed in any way. 
\item Change the CDT: Inside the \texttt{ChangeGenerator.generateChange}, the 
\texttt{NodeCommentMap} of the translation unit is fetched. By writing a patch, 
it was possible to insert new mappings into this map. This allowed to move 
comments of an old node to any newly created node. 

Pro: automatic indentation, developer may choose where to put the comments, 
every comment may be preserved. 

Contra: does not deal with macros, five classes need to be changed in CDT, 
comments need to be moved by hand. See the branch 'inject' inside the repository 
to study this solution.
\item Find and insert comments by hand using an IASTComment. 

Pro: lets the developer decide where to put the comment. 

Contra: Feature is commented-out in the 7.0.1 release of CDT, comments need to 
be moved by hand.
\item Other solutions may be possible. An idea could be to register the comments 
during copy functions. This would require to change every copy function of every 
IASTNode. 
\end{itemize}

\subsubsection{Toggling function-local functions}
\textbf{Problem}: When function-local functions are allowed to be toggled, the 
fuzzy selection detection may not be as intuitive to the user as intended. If 
the cursor is inside a function body, the parent function definition should be 
toggled.

\textbf{Cause}: At least GCC allows to define functions inside a function, even 
templated or namespaced ones or ones that are inside a class definition. In the 
beginning, selection detection just found the nearest definition around the 
selection.

\textbf{Solution}: Function-local functions are skipped in favour of the next 
parent that is a declaration that may be toggled. If no parent is found, the 
refactoring aborts.


\section{Testing and Performance}

This section introduces special tricks that were used to simplify testing and
control performance.

\subsection{A real-world test environment}
The COAST source code uses modern C++ coding style and was hence used as test 
environment for real-world tests.

\subsection{Fuzzy whitespace recognition for the test environment}

As described in past theses at HSR, the refactoring testing environment
needed an exact definition of the generated code. This was annoying because
same-looking code samples could result in a red bar if white spaces were not the
same. To make writing new tests easier, the comparison method was overridden to
support fuzzy whitespace recognition.

Leading tabs or whitespaces are recognized and it is assumed that the same
indention is used for the whole file. In addition, trailing newlines that are
added by the ASTRewriter are ignored.

The changes made to the CDT test environment help writing new tests without
having to care whether the ASTRewriter uses spaces or tabs for indention.
Resulting code that looks the same now gives a green bar.

\subsection{Testing issues}

The whitespace issue was already discussed. Another issue is that the Doxygen 
syntax \/\/\! may not be used in the test files since this syntax is used to 
specify the test name.

Testing error conditions is not comfortable. % TODO: solutions?

All in all, the special refactoring test environment was a big help for 
refactoring relaxedly.

\subsection{Performance tests}

The simplest way to assess the speed of the refactoring is to look at the junit 
time measurements. The first test that is run takes more time and represents the 
time needed for first time toggling when the refactoring infrastructure has to 
be loaded. 

All performance tests must be executed on the same developer machine, taking the 
average time of three consecutive runs of all tests. Four scenarios have been 
chosen to be able to observe the performance of the toggle refactoring:

\begin{enumerate}
\item First time toggling: includes loading of the infrastructure and will take 
some more time.
\item Toggle from in class to header: only one translation unit is affected by 
this refactoring.
\item Toggle from implementation header: two translation units are affected here.
\item Average runtime: time to run all the tests (will vary due to new tests but 
helpful for comparing effects of small changes between versions)
\item Emtpy reference test: a dummy refactoring that won't load and analyze any 
code. Shows what amount of time is consumed by the given refactoring 
infrastructure.
\end{enumerate}

Another technique to measure time more accurately was checked out. For this, the 
org.eclipse.test.performance plugin was used. 

\subsubsection{Results}

It is difficult to compare the speed to other refactorings of CDT since wizards 
are used for the other known refactorings. However, the goal was reached that 
the refactoring is executing almost instantly.

The following table shows speed improvements during the development of the 
project:

\begin{tabular}[t]{l|rrr}
 Scenario   & first draft & final result & improvement in \% \\
 \hline
 first time toggling 	&  ---ms & --ms & --\% \\
 toggle free function	&  ---ms & --ms & --\% \\
 average runtime	&  ---ms & --ms & --\% \\
 emtpy reference test	&  ---ms & --ms & --\% \\
\end{tabular}

The results from the \textit{org.eclipse.test.performance} speed tests were not 
used in the end. Since in reality the refactorings are much slower than the 
(repeated) measurements, resulting values may only be compared relative to each 
other.

