\documentclass[a4paper,12pt]{scrreprt}
\usepackage[utf8x]{inputenc}
\usepackage{graphicx}

\usepackage{listings}
\usepackage{color}
\lstset{language=C++, numbers=left,
numberstyle=\tiny\color{black},frame=shadowbox, basicstyle=\small,
rulesepcolor=\color{gray}, backgroundcolor=\color{lightgray}, title=\lstname,
captionpos=b}
\definecolor{gray}{rgb}{0.7,0.7,0.7}
\definecolor{lightgray}{rgb}{0.95,0.95,0.95}

\textheight 20cm
\textwidth 12.0cm
\headsep 2.5cm
\oddsidemargin 2.0cm
\evensidemargin 2.0cm

\usepackage{fancyhdr}
\pagestyle{fancy}
\fancyhead{}
\fancyfoot{}
\rhead{\leftmark}
\rfoot{\thepage}
\lfoot{\today}

\title{One touch C++ code automation for Eclipse CDT \\ ~ \\ 
\normalsize{Toggle Member Function} }
\titlehead{\includegraphics{logo_hsr_farbig.pdf}}
\subject{semester thesis}
\date{\today}
\publishers{Supervisor: Prof. Peter Sommerlad}
\author{Martin Schwab, Thomas Kallenberg}
\begin{document}

\maketitle
\pagenumbering{roman}

\begin{abstract}
\thispagestyle{empty}
During this semester thesis, a code automation tool has been developed for the 
Eclipse C++ Development Toolkit (CDT) using the Eclipse refactoring mechanism. 
The resulting plugin enables a C++ developer to move function definitions easily 
between header and source files. 

The new plugin differs from existing plugins in the way that it minimizes human 
interaction by providing a single keystroke interface. The refactoring gets by 
with no user wizard at all and is tolerant to imprecise code selection. 

This document discusses the uses of the plugin as well as the problems that had 
to be solved during the project. Students developing a new refactoring may have 
a look at the problems section to be able to start with their own project 
quickly.
\end{abstract}

\chapter*{Management Summary}
\thispagestyle{empty}

\tableofcontents
\thispagestyle{empty}
\pagenumbering{arabic}

\chapter{Introduction}
\thispagestyle{fancy}

\section{The current situation}

The Eclipse Java Development Toolkit (JDT) has a large set of both quick and
useful refactorings. Its sibling the C++ Development Toolkit (CDT) offers just a
small range of such code helpers today. In addition, some of them don't work
satisfyingly: Currenty, extracting the body of a hello world function takes more
than three seconds on our machines. Reliability? Try to \texttt{extract
constant} the hello world string of the same program. Currently it will fail.
The code won't even compile.\newline
Every bachelor student at HSR should to visit a C++ class. Nearly
everybody uses Eclipse to solve the exercises. After a while it was clear that
touching the refactoring buttons was a dangerous action because in some cases
descibed above they broke your code. Compile errors all over the place and
difficult exercise assignments didn't make our life easier.\newline
One annoying problem in C++ is the separation of the source and the header
files. This is a pain point for every programmer. Forgetting to update the
function signature in one of the files will result in compilation error which
causes either lack of understanding for beginners or loss of time.\newline
After two minutes of compile error hunting because you forgot to rename a
function signature in both, the header and the implementation file, you start
asking yourself: Why nobody has yet implemented a solution to prevent such an
error?\newline
One answer seems to be that this is not an easy task to achieve. But
nevertheless we do not want to be part of the people who always complain about
open source software. We change it, because we can. \newline
 
The IFS Institute for Software at HSR with its group around Professor Peter
Sommerlad and
Emanuel Graf has been working on Eclipse refactorings for a long time. Since
2006 nine Eclipse refactoring projects have been completed.\newline

\section{What is planed to do}

During this semester thesis it is planned to introduce and improve one or more
refactorings to the Eclipse CDT project. The priorities are as follows

\begin{enumerate}
\item Toggle (Member-) Function Implementation
\item Re-implement Implement Function
\item Override virtual Member Function
\end{enumerate}

There is a need for three toggling subfunctions to enable toggling circularly
from \texttt{in-class}
to \texttt{in-header} to \texttt{separate-file}  and back again to
\texttt{in-class}. We concentrate on this order during this project. If the
implementation is fast enough there is no need for a configuration dialog which
defines the order of the toggle subfunctions.\newline

Depending on the success of the implementation with those three variations of
the
first refactoring we continue with the re-implementation of  ``Implement
Function'' and then ``Override virtual member function''.\newline

The re-implementation of ``implement function'' should be very fast. It should
only create an empty block below the function signature. This newly created
block can then be toggled to the implementation file.

\section{Basic Goals}

\begin{itemize}
 \item Toggling between \texttt{in-class}, \texttt{in-header},
\texttt{separate-file} and back again to \texttt{in-class} works for basic and
some frequent special cases.
 \item Project organization: Fixed two-week iterations are used. Redmine is used
for planning and tracking time, issue tracking and as information radiator for
the supervisor. A project documentation is written. Organization and results are
reviewed weekly together with the supervisor.
 \item Quality: Common cases are covered with test cases for each
refactoring subtype.
 \item Integration and Automation: Sitting in front of a fresh Eclipse CDT
installation a first semester student can install our refactoring using an
update site as long as the functionality is not integrated into the main CDT
plug-in.
 \item To minimize the integration overhead with CDT it will be worked closely
with Emanuel Graf as he is a CDT commiter.
 \item At the end the project will be handed to the supervisor with two CD's and
two paper versions of the documentation. An update site is created where the
functionality can be added to Eclipse. A website describes in short the
functionality and our project vision.
\end{itemize}

\section{Advanced goals}
All basic goals have been archived. Additionally:\newline
\begin{itemize}
 \item Toggling function is fast. less than 1s
\end{itemize}

Re-Implement the ``Implement Function'' feature.
\begin{itemize}
 \item A new function block is created with nearly no delay right below the
function signature.
 \item A default return statement is created when the block is created.
 \item If the return statement cannot be determined, a comment is inserted into
the block.
\end{itemize}

\section{Expected outcome}

Implement member function and the toggle key are written to work in synergy.
First write the declaration for a member function in the class definition, then
a hot-key is used to implement the function. At this point the toggle key can
be hit at any time to move the function to the appropriate position and
continue with the next new member function.

\section{High level goals and outlook}

If there is enough time an "Override virtual function" is implemented.
Additionally, content assist can be implemented. This could be part of a
bachelor thesis which continues and completes the work done in this semester
thesis.

\section{Project duration}
The semester thesis starts on September 20th and has to be finished until
December 23rd, 2010.
\chapter{Specification}
\thispagestyle{fancy}

This section describes how the different code automation mechanisms have been
understood and designed.

\section{\textit{Toggle Function Definition}}

The idea: good code should separate interface and implementation. However, it is 
annoying to copy function signatures from the header file to the implementation 
file or vice versa. 

\textit{Toggle Function Definition} moves the code of a member function between 
different possible places, preserving the declaration statement inside the 
header file. What the different places are, in which direction the code may be 
moved and in which situation the refactoring may be invoked is described in the 
following chapters. 

\subsection{Examples}

\subsubsection{Toggle free function from .cpp to .h}
If \marginline{declaration exists} a separate declaration of the function exists,
the declaration shall be substituted by the definition. If the definition is 
placed inside a header file, its declaration shall remain there. This implies 
that when the following code 

\begin{lstlisting}[caption={A.h},label={01freefuncPre},language=C++]
#ifndef EXAMPLE_H
#define EXAMPLE_H

class ClearClass {
  void bigfunction() {
    /* implementation */
  }
};

#endif
\end{lstlisting}

is toggled twice, it will result in a declaration inside the header file.

What happens if no declaration is found? \marginline{no declaration}
Whenever a free function definition is toggled the first time, the whole 
definition shall be moved to a header file with the same name as the source 
file. When this definition is toggled back to the source file, the declaration will remain in
Whenever a free function is toggled, it shall swap definition and declaration. 
When toggling to the header file and no declaration may be found there, the 
definition will be created there.
with the same name as the source file
\begin{tabular}{p{5cm}p{.4cm}p{5cm}}
\begin{lstlisting}[caption={A.h},label={01freefuncPostCpp},language=C++]
void freefunction {
  return;
}
\end{lstlisting}
& & 
\begin{lstlisting}[caption={A.cpp},label={01freefuncPostH},language=C++]
#ifndef FREEFUNC_H_
#define FREEFUNC_H_

void freefunc() {
  return;
}

#endif
\end{lstlisting}
\end{tabular}

\subsection{Activation}

Most important is this basic rule: A function definition must exist.

The refactoring may be activated as soon as the user is editing a C++ 
translation unit and the cursor resides inside a function declaration or 
definition. For the selected function, a function definition must exist in an 
associated file. If no definition exists, the refactoring aborts. There may be 
more than one declaration. If more than one declaration is selected, it is not 
specified which one will be toggled.

The refactoring shall allow selections anywhere inside the function, be it 
inside the signature, function (try) body, catch handlers or template 
declarations. 

\subsection{Three positions for function definitions}

\label{positions}
In C++ there are three possible positions where a \marginline{\textit{in-class}}
function definition may occur. Listing \ref{classheaderimpl} shows an example
where the definition of a class member function is placed inside its class
definition. Placing implementation code right next to the declaration is the
most intuitive behaviour for Java developers. New code blocks created by
\textit{Implement Member Function} are placed inside the class definition too.

\lstinputlisting[caption={In Class implementation in A.h},
label={classheaderimpl}]{classheader_implementation.h}

%TODO: add a footnote for the 'extern' problem. (MS, 2.12.2010)
To keep the interface clear, \marginline{\textit{in-header}} functions may be 
placed outside the class definition but still in the same (header) file. Such a 
function is called \textit{inlined}. For templates, this is the only position 
where implementation may be placed outside the class definition due to problems 
of the \textit{extern} keyword. Templated functions cannot be placed outside the 
header file. Except for ... %TODO: find cases where template functions can be
%placed outside of a header file
The following listings show an example of what will be called 
\textit{in-header situation} throughout this document.

\lstinputlisting[caption={Implementation outside of the class in A.h},
label={outsideclass}]{outside_class_implementation.h}

To \marginline{\textit{in-file}} separate implementation from the interface more 
clearly, a separate source file may be used for the definitions while the 
declarations remain in the header file. An example for this position of a 
function definition is shown by listings \ref{twofilesolution_header} and
\ref{twofilesolution_impl}. The position will be called 
\textit{in-implementation} or \textit{in-file}.

\vspace{0.5cm}
\begin{minipage}{.48\textwidth}
\lstset{xrightmargin=0.5cm}
\begin{lstlisting}[caption={A.h},label={twofilesolution_header},language=C++]
#ifndef A_H_
#define A_H_

class A {
	int x();
};

#endif /* _A_H */
\end{lstlisting}
\end{minipage}%
\begin{minipage}{.48\textwidth}
\lstset{xleftmargin=0.5cm}
\begin{lstlisting}[caption={A.cpp},label={twofilesolution_impl},language=C++]
#include "A.h"

int A::x() { 
    /* impl. */ 
}
\end{lstlisting}
\end{minipage}

\subsection{Basic scenarios}

Depending on the current selections, different strategies need to be applied to 
move the function definition. All supported toggling situations and their 
special cases are listed in this chapter.

\subsubsection{Free functions}
For plain free functions, toggling shall happen between two positions:
\begin{enumerate}
\item Toggle from in-header to in-file
\item Toggle from in-file to in-header
\end{enumerate}

\subsubsection{Member functions}
For functions inside classes, toggling is expected to be available for three 
positions:
\begin{enumerate}
\item Toggle from in-class to in-header
\item Toggle from in-header to in-file
\item Toggle from in-file to in-class
\end{enumerate}

\subsection{Special cases}

Not every function may be toggled between the three positions and some cases 
require additional work before they may be toggled. Those special case are 
listed in this section.

\subsubsection{Namespaces}

If the moved function definition is contained inside a namespace definition, a 
new namespace definition shall be created. The function is then inserted into 
the newly created namespace definition.

Contrariwise, namespace definitions that would become empty after removing the 
last function shall be deleted.

\subsubsection{Templated member functions}

An exception are templated member functions that may only be toggled inside the 
same header file.
\begin{enumerate}
\item Toggle from in-class to in-header
\item Toggle from in-header to in-class
\end{enumerate}

\section{\textit{Implement Member Function}}

\subsection{Activation}
This refactoring shall be active as soon as a function declaration is selected 
that has no associated definition. A missing semicolon at the end of the 
function declaration should not stop the mechanism to work.

After writing a function declaration in the class definition, not yet written
the ``;'', code completion can be used to create the function body with an
appropriate default empty return statement.\newline

An already completed function declaration can be transformed to a function
definition by using the ``Implement Member Function'' hot-key which creates a
body with default empty return statement.

\subsection{Expected result}
As described above, functions may only be toggled when they provide a function 
body. This refactoring shall provide a facility to create an empty function body 
with a default return value to enable \textit{Toggle Function Definition}.

The re-implementation of \textit{implement function} must be very fast. 

\section{Override virtual member function}

No deeper investigation has been done for this refactoring since it was not 
implemented during the project.


\chapter{Implementation and Solution}
\thispagestyle{fancy}

\section{Architecture}

In Eclipse, most of the architecture is already given. Some specialities of the 
toggle refactoring are presented in this section.

\subsubsection{Basic call flow}
Here's the place for some nice diagram to visualize the work.

Activator -- ActionDelegate -- (Job) -- ToggleRefactoring -- Strategy

\subsubsection{The strategies}
The \textit{ToggleStrategyFactory} is used to decide which strategy should be 
considered, based on the user selection. The following strategies have been 
implemented to cover the specified cases:

\begin{itemize}
\item ToggleFromImplementationToClassStrategy
\item ToggleFreeFunctionFromInHeaderToImpl
\item ToggleFromClassToInHeaderStrategy
\item ToggleFromInHeaderToClassStrategy
\item ToggleFromInHeaderToImplementationStrategy
\end{itemize}

\subsubsection{implications of not using a refactoring wizard}
No wizard was used for this refactoring since it must be fast and may be 
executed several times in succession. When using a wizard, the 
\textit{RefactoringWizardOpenOperation} handles the execution of the refactoring 
inside a separate job. Since the toggle refactoring does not use the wizard, a 
separate job had to be scheduled by the ActionDelegate.

In addition, the undo functionality had to be implemented separately. When the 
changes are performed, they (surprisingly) also return the undo changes that are 
needed by the UndoManager. The UndoManager is available through 
\textit{RefactoringCore.getUndoManager()}. See \textit{ToggleRefactoringRunner} 
for more details on the implementation.

\section{Testing and Performance}

This section introduces special tricks that were used to simplify testing and
control performance.

\subsection{Fuzzy whitespace recognition for the test environment}

As described in past theses at HSR, the refactoring testing environment
needed an exact definition of the generated code. This was annoying because
same-looking code samples could result in a red bar if white spaces were not the
same. To make writing new tests easier, the comparison method was overridden to
support fuzzy whitespace recognition.

Leading tabs or whitespaces are recognized and it is assumed that the same
indention is used for the whole file. In addition, trailing newlines that are
added by the ASTRewriter are ignored.

The changes made to the CDT test environment help writing new tests without
having to care whether the ASTRewriter uses spaces or tabs for indention.
Resulting code that looks the same now gives a green bar.

\subsection{Performance tests}

Performance was tested using the org.eclipse.test.performance plugin. Four
different scenarios have been chosen for comparison:

\begin{enumerate}
\item testWithIncludeStatements()
\item testWithoutIncludeStatements()
\item testInClassToInHeader()
\item testInHeaderToInClass()
\end{enumerate}

In addition, an all-together test has been included for quick performance
comparison and a reference test that did nothing was run to measure the overhead
of the performance test framework. Since in reality the refactorings are slower
than the (repeated) measurements, resulting values should be considered relative
to each other. The same developer laptop was used for before and after tests.

\begin{tabular}[t]{l|rrr}
 Scenario   & first draft & final result & improvement in \% \\
 \hline
 Scenario 1	&  364ms & --ms & --\% \\
 Scenario 2	&  337ms & --ms & --\% \\
 Scenario 3	&  335ms & --ms & --\% \\
 Scenario 4	&  323ms & --ms & --\% \\
 All tests	& 5530ms & --ms & --\% \\
 Reference test	&    0ms & --ms & --\% \\
\end{tabular}

\subsection{Issues}

This section describes special cases that have been omitted intentionally or due 
to lack of time. In addition, found limitations of the CDT are described here.

\subsubsection{Constructor/Destructor bug}
\textbf{Problem}: Let CDT create a new class with a constructor and a destructor. 
Then toggle the constructor out of the class definition. The Destructor will be overridden partially. This problem only occurs in exactly this situation (no paramenters, no initialization lists).
\textbf{Cause}: Unknown. It seems to be some offset bug.
\textbf{Solution}: None yet.

\subsubsection{Menu integration}
\textbf{Problem}: Adding a new menu item to the refactor menu is difficult when 
developing a separate plugin.
\textbf{Cause}: Menu items are hardcoded inside 
\textit{CRefactoringActionGroup}. No way was found to replace or change this 
class within a separate plugin. In addition, the use of the 
\textit{org.eclipse.ui.actionSets} extension point does not make inserting new 
items easier.
\textbf{Solution}: The menu was added using \textit{plugin.xml} and may be added 
by the user manually. Right-click the toolbar, choose "Customize Perspective...", 
"Command Groups Availability" and check every group that is named "C++ Coding". 
This reveals the new menu item inside the refactor menu. Anyhow, the refactoring 
may always be invoked using the key binding.

The toggle key binding was realized using the \textit{org.eclipse.ui.bindings} 
extension point.

\subsubsection{The selection}
\textbf{Problem}: After toggling multiple times, the wrong functions were 
toggled or no selected function was found at all. 
\textbf{Cause}: The region provided by \textit{CRefactoring} pointed to a wrong 
code offset. 
\textbf{Solution}: The current selection is now based directly on the current 
active editor part's selection.

\subsubsection{Comments and macros}
\textbf{Problem}: Nodes inside a translation unit have to be copied to be 
changed since they are frozen. When nodes are copied, their surrounding comments 
get lost during rewrite. This was annoying, since copying the function body 
provided a straightforward solution for replacing a declaration with a 
definition.

Another issue were macros. Macros are working perfectly when copied and 
rewritten inside the same translation unit. As soon as a macro is moved outside 
to another translation unit, the macro will be expanded during rewrite. 

\textbf{Cause}: The rewriter is using a method in \texttt{ASTCommenter} to get a 
\texttt{NodeCommentMap} of the rewritten translation unit. If a node is copied, 
it has another reference which won't be inside the comment map anymore. Thus, 
when the rewriter writes the new node, it won't notice that the node was 
replaced by another.

\textbf{Solutions}:
\begin{itemize}
\item Get the raw signature of the code parts that should be copied and insert 
them using an ASTLiteralNode. 

Pro: It works without changing the CDT core and macros are not expanded. 

Contra: Breaks indentation and inserts unneeded newlines. This solution was used 
finally because whitespace issues may be dealt with the formatter.
\item Do as \texttt{ExtractFunction} does: rewrite each statement inside the 
function body separately. 

Pro: automatic indentation. 

Contra: touches the body although it does not need to be changed in any way. 
\item Change the CDT: Inside the \texttt{ChangeGenerator.generateChange}, the 
\texttt{NodeCommentMap} of the translation unit is fetched. By writing a patch, 
it was possible to insert new mappings into this map. This allowed to move 
comments of an old node to any newly created node. 

Pro: automatic indentation, developer may choose where to put the comments, 
every comment may be preserved. 

Contra: does not deal with macros, five classes need to be changed in CDT, 
comments need to be moved by hand. See the branch 'inject' inside the repository 
to study this solution.
\item Find and insert comments by hand using an IASTComment. 

Pro: lets the developer decide where to put the comment. 

Contra: Feature is commented-out in the 7.0.1 release of CDT, comments need to 
be moved by hand.
\item Other solutions may be possible. An idea could be to register the comments 
during copy functions. This would require to change every copy function of every 
IASTNode. 
\end{itemize}

\subsubsection{Toggling function-local functions}
\textbf{Problem}: When function-local functions are allowed to be toggled, the 
fuzzy selection detection may not be as intuitive to the user as intended. If 
the cursor is inside a function body, the parent function definition should be 
toggled.

\textbf{Cause}: At least GCC allows to define functions inside a function, even 
templated or namespaced ones or ones that are inside a class definition. In the 
beginning, selection detection just found the nearest definition around the 
selection.

\textbf{Solution}: Function-local functions are skipped in favour of the next 
parent that is a declaration that may be toggled. If no parent is found, the 
refactoring aborts.



\chapter{Interpretation}
\thispagestyle{fancy}

\subsection{Personal review}

\subsubsection{Martin Schwab}

\subsubsection{Thomas Kallenberg}

\chapter{Conclusions and future work}
\thispagestyle{fancy}

\chapter{Sources}
\thispagestyle{fancy}
A. Verhein, A. Simeon, "Werkzeugkasten Technische Berichte 1", 31.07.2008

\chapter*{User agreement}
\chapter*{Copyright agreement}
\part{Appendix}
\chapter{Project setup}
\thispagestyle{fancy}

Project setup needed two weeks to become stable. In this chapter, solutions are 
described that may help future projects to build up their project environment 
more efficiently.

\section{Configuration management}

Configuration management always seems to be the same. This section describes 
what techniques were used throughout this project and what product versions were 
needed to realize an agile production environment for Eclipse plug-in
development.

\subsection{Project Server}

As a project Server Ubuntu Linux 10.04.1 LTS (Lucid Lynx) on a VmWare Cluster
was used.

\subsection{Git}

As a version control system, git was used. This time it was not used like an SVN
replacement but instead to get some redundancy by storing the source code on
multiple servers. Both project developers had their ``own`` git server on which
the developer committed. It was merged between these servers and then pushed to
the main repository for automatic testing.

Later in the project, a master and a development branch have been introduced to 
improve trust for the master branch.

\subsection{Maven and Tycho}

Tycho is a plug-in for Maven to build an Eclipse plug-in and to execute its tests.
Maven3 is required for this to work. Although Maven3 is beta, it was proven
stable during the project.

\subsection{Hudson}

Hudson is a build server for executing repeated build jobs. It is specialized
for executing Java projects. To get the tests executed on the fake X sever, the
\texttt{DISPLAY} environment variable must be set. If not, tests will fail with
a cryptic SWTError.

There is a plug-in for Hudson to set the environment variable to
the right value. In most cases this is \textbf{:0}.

\section{Project management - Redmine}

Redmine was used to track issues, milestones and agendas and as an information 
radar for the supervisor.

The redmine version used crashed every now and then. This issue went away after
some memory upgrade. Redmine depends how it is configured. Using the passenger
ruby module made it quite stable. 

\section{Test environment}

To execute the tests on a headless server with Hudson build server first a fake
X Server was needed. Xvfb was used for the job. Maven has to be
explicitly told which test class to execute and in which folder the test class
is located. If this is not done properly the tests will fail.

\subsection{Test coverage}

Creating new files was not tested in the beginning due to the dialog box popping 
up during execution. However, with the help of the same technique that was used 
to test for errors, the problem could be solved. Before a dialog box is popped 
up asking whether to create a new file, it is checked whether the test framework 
has set a flag to skip that question.

The code coverage tool EclEmma was a good help to find dead code and to think
about justification of code blocks.

\subsection{Documentation}

The documentation was originally based on \cite{AV08} and adapted to meet the 
requirements of the project.


\chapter*{User Manual}
\chapter*{Time managing}
\chapter*{Glossary}

\thispagestyle{fancy}
%\section{Poster}

\end{document}          
