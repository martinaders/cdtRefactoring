\documentclass[a4paper,12pt]{scrreprt}
\usepackage[utf8x]{inputenc}
\usepackage{graphicx}

\usepackage{listings}
\usepackage{color}
\lstset{language=C++, numbers=left,
numberstyle=\tiny\color{black},frame=shadowbox, basicstyle=\small,
rulesepcolor=\color{gray}, backgroundcolor=\color{lightgray}, title=\lstname,
captionpos=b}
\definecolor{gray}{rgb}{0.7,0.7,0.7}
\definecolor{lightgray}{rgb}{0.95,0.95,0.95}

\textheight 19cm
\textwidth 14cm
\headsep 2.5cm
\oddsidemargin 0.75cm
\evensidemargin 0.75cm

\usepackage{fancyhdr}
\pagestyle{fancy}
\fancyhead{}
\fancyfoot{}
\rhead{\leftmark}
\rfoot{\thepage}

\title{CDT Refactorings}
%\subtitle{Toggle member function, Implement function, Override virtual member function}
%\titlehead{\includegraphics{logo_hsr_farbig.pdf}}
\subject{semester thesis}
\date{\today}
\publishers{Supervisor: Peter Sommerlad}
\author{Martin Schwab, Thomas Kallenberg}
\begin{document}

\maketitle
\pagenumbering{roman}

\begin{abstract}
Welches Problem wurde mit welchen Ergebnissen bearbeitet + Schlussfolgerungen.
\end{abstract}

\tableofcontents
\thispagestyle{fancy}
\pagenumbering{arabic}

\chapter{Introduction}
\thispagestyle{fancy}

\section{The current situation}

The Eclipse Java Development Toolkit (JDT) has a large set of both quick and
useful refactorings. Its sibling the C++ Development Toolkit (CDT) offers just a
small range of such code helpers today. In addition, some of them don't work
satisfyingly: Currenty, extracting the body of a hello world function takes more
than
three seconds on our machines. Reliability? Try to \texttt{extract constant} the
hello world string of the same program. Currently it will fail. The code won't
even
compile.\newline
Every bachelor student at HSR should to visit a C++ class. Nearly
everybody uses Eclipse to solve the exercises. After a while it was clear that
touching the refactoring buttons was a dangerous action because in some cases
descibed above they broke your code. Compile errors all over the place and
difficult exercise assignments didn't make our life easier.\newline
One annoying problem in C++ is the separation of the source and the header
files. This is a pain point for every programmer. Forgetting to update the
function signature in one of the files will result in compilation error which
causes either lack of understanding for beginners or loss of time.\newline
After two minutes of compile error hunting because you forgot to rename a
function signature in both, the header and the implementation file, you start
asking yourself: Why nobody has yet implemented a solution to prevent such an
error?\newline
One answer seems to be that this is not an easy task to achieve. But
nevertheless we do not want to be part of the people who always complain about
open source software. We change it, because we can. \newline
 
The IFS Institute for Software at HSR with its group around Professor Peter
Sommerlad and
Emanuel Graf has been working on Eclipse refactorings for a long time. Since
2006 nine Eclipse refactoring projects have been completed.\newline

\section{What is planed to do}

During this semester thesis it is planned to introduce and improve one or more
refactorings to the Eclipse CDT project. The priorities are as follows

\begin{enumerate}
\item Toggle (Member-) Function Implementation
\item Re-implement Implement Function
\item Override virtual Member Function
\end{enumerate}

There is a need for three toggling subfunctions to enable toggling circularly
from \texttt{in-class}
to \texttt{in-header} to \texttt{separate-file}  and back again to
\texttt{in-class}. We concentrate on this order during this project. If the
implementation is fast enough there is no need for a configuration dialog which
defines the order of the toggle subfunctions.\newline

Depending on the success of the implementation with those three variations of
the
first refactoring we continue with the re-implementation of  ``Implement
Function'' and then ``Override virtual member function''.\newline

The re-implementation of ``implement function'' should be very fast. It should
only create an empty block below the function signature. This newly created
block can then be toggled to the implementation file.

\section{Basic Goals}

\begin{itemize}
 \item Toggling between \texttt{in-class}, \texttt{in-header},
\texttt{separate-file} and back again to \texttt{in-class} works for basic and
some frequent special cases.
 \item Project organization: Fixed two-week iterations are used. Redmine is used
for planning and tracking time, issue tracking and as information radiator for
the supervisor. A project documentation is written. Organization and results are
reviewed weekly together with the supervisor.
 \item Quality: Common cases are covered with test cases for each
refactoring subtype.
 \item Integration and Automation: Sitting in front of a fresh Eclipse CDT
installation a first semester student can install our refactoring using an
update site as long as the functionality is not integrated into the main CDT
plug-in.
 \item To minimize the integration overhead with CDT it will be worked closely
with Emanuel Graf as he is a CDT commiter.
 \item At the end the project will be handed to the supervisor with two CD's and
two paper versions of the documentation. An update site is created where the
functionality can be added to Eclipse. A website describes in short the
functionality and our project vision.
\end{itemize}

\section{Advanced goals}
All basic goals have been archived. Additionally:\newline
\begin{itemize}
 \item Toggling function is fast. less than 1s
\end{itemize}

Re-Implement the ``Implement Function'' feature.
\begin{itemize}
 \item A new function block is created with nearly no delay right below the
function signature.
 \item A default return statement is created when the block is created.
 \item If the return statement cannot be determined, a comment is inserted into
the block.
\end{itemize}

\section{Expected outcome}

Implement member function and the toggle key are written to work in synergy.
First write the declaration for a member function in the class definition, then
a hot-key is used to implement the function. At this point the toggle key can
be hit at any time to move the function to the appropriate position and
continue with the next new member function.

\section{High level goals and outlook}

If there is enough time an "Override virtual function" is implemented.
Additionally, content assist can be implemented. This could be part of a
bachelor thesis which continues and completes the work done in this semester
thesis.

\section{Project duration}
The semester thesis starts on September 20th and has to be finished until
December 23rd, 2010.

\chapter{Concepts and Design}
\thispagestyle{fancy}

\section{Toggle Key}

\subsection{The different position of definition}

In C++ there are three possible positions where a \marginline{definition
position} function definition may occur. Listing \ref{classheaderimpl} shows
the definition in the header. This is a rather uncommon for a definition but the
initial position placed by the new \textit{Implement Member Function} described
in later in 1.1.2.
%TODO: add right link here
\lstinputlisting[caption={In Class implementation in A.h},
label={classheaderimpl}]{classheader_implementation.c}

The most common position for ``templated`` member functions are described by
Listing \ref{outsideclass}. Due to the problem of the \textbf{extern} TODO:LINK
keyword the definition must stay in the header file.
\lstinputlisting[caption={Implementation outside of the class in
A.h}, label={outsideclass}]{outside_class_implementation.c}

The most common way in C++ is the separation of header and the implemenation
file, shown by listing \ref{twofilesolution_header} and listing
\ref{twofilesolution_impl}.

\begin{tabular}{p{5cm}p{.5cm}p{6cm}}
\lstinputlisting[caption={two file solution - A.h},
label={twofilesolution_header}]{implementation_file.h}
& & 
\lstinputlisting[caption={two file solution - A.cpp},
label={twofilesolution_impl}]{implementation_file.c}
\end{tabular}

\subsection{How it works}

The \textit{Toggle Key} is used to move the definition of a member function
between the positions described above. For a non-templated member function
moving will be done in the following order:
\begin{itemize}
 \item From inline in the class in the header file
 \item to the outside of the class in the same header file
 \item to the implementation file
 \item and back to inline of the class in the header file.
\end{itemize}

However, for templated member function this does not make much sense. Therefore
templated functions will only have a two way toggle.
\begin{itemize}
 \item From inline in the class in the header
 \item to the outside of the class in the same header
 \item and then back to inline in the class.
\end{itemize}

The Editor is not switched at any time but stays at the same file.

\section{Re-Implementation of Implement Member Function}

\subsection{The need for a new Implement Member Function}

The current CDT plug-in already includes a \textit{Implement Member
Function}. But this implementation is slow and it does not really
fit together with the \textit{Toggle Key}. It breaks the coding
flow for adding functionality to classes twice which could be reason not to use
the toggle key. Therefore toggle key will be supported by a new
Implement Member Function which synergies with the toggle key.

\subsection{How it works}

After writing a function declaration in the class definition, not yet written
the ``;'', code completion can be used to create the function body with a
appropriated default empty return statement.\newline

An already completed function declaration can be transformed to a function
definition by using the ``Implement Member Function'' hot-key which creates a
body with default empty return statement.

\section{Override virtual member function}

\chapter{Implementation}
\thispagestyle{fancy}
This chapter describes how the solution was actually implemented.

\chapter{Tests and Results}
\chapter{How we tested, metrics}
\chapter{Results}
\chapter{Interpretation}

%\section{Projektbasis}
%\section{Bemessung}
%\section{Zeichnungen}
%\section{Diverses}

\chapter{Conclusions and future work}
Results, Bewertung, open issues, Empfehlungen

\chapter{Sources}
A. Verhein, A. Simeon, "Werkzeugkasten Technische Berichte 1", 31.07.2008

\chapter*{Nutzungsvereinbarung}
\chapter*{Erklärung der Urheberschaft}
\part{Appendix}
\thispagestyle{fancy}
%\section{Poster}


\end{document}          
